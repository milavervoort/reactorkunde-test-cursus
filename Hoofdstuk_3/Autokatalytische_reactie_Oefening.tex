\documentclass{ximera}

\title{Autokatalytische reactie: Oefening}
\author{Mila Vervoort}
\license{CC: 0}         % replace with an appropriate license, or set it in xmPreamble

\begin{document}
\begin{abstract}
    Oefening op autokatalytische reactie
\end{abstract}
\maketitle

\section{Oefening: autokatalytische reacties}

We beschouwen een autokatalytische reactie van de vorm

\[
A \rightarrow B
\]

met reactiekinetiek

\[
r_A = -k C_A C_B
\]

Hoe kunnen we deze autokatalytische reacties drijven tot een zeer hoge conversie? 

We veronderstellen:
\begin{itemize}
\item isotherme werking
\item ideale reactoren
\item constante densiteit
\end{itemize}


\subsection*{Gegeven}

\begin{align*}
C_{A0} &= 1 \ \text{kmol/m}^3 \\
C_{B0} &= 0 \\
q &= 10 \ \text{m}^3/\text{h} \\
C_{A,\text{uit}} &= 0.01 \ \text{kmol/m}^3 \\
k &= 4.2 \times 10^{-4} \ \text{m}^3/(\text{kmol}\cdot \text{s})\\
X_A &= 0.99\\
\end{align*}

\subsection*{Gevraagd}

\begin{enumerate}
\item Bepaal het benodigde volume van één CSTR.
\item Bepaal het benodigde volume voor een serieschakeling van: één CSTR gevolgd door één PFR met een intermediaire concentratie
  \[
  C_A = 0.5 \ \text{kmol/m}^3.
  \]

\item Bepaal de optimale recycleverhouding en het vereiste volume voor een PFR met recycle.
\end{enumerate}

\subsection*{Eén CSTR}

\begin{exercise}
\begin{question}
Bepaal het benodigde volume van één CSTR. 
\[
\text{Volume} = \answer{661} m^3
\] 
\begin{hint}
    Schrijf de stofbalans voor een CSTR uit. De stof balans is: 
    \[ 
    Q*C_{A,in} - Q*C_{A,out} + R_A*V = 0
    \]
\end{hint}
\begin{hint}
    Vul de reactiekinetiek voor een CSTR in.
    \[
    -R_A = k C_A (C_{A0} - C_{A}) 
    \]
\end{hint}
\begin{feedback}
    Voor een autokatalytische reactie:
    \[
    -r_A = k C_A (C_{A0} - C_A)
    \]
    Invullen in het ontwerpverband:
    \[\tau =
    \frac{C_{A0} - C_{A,\text{uit}}}
    {k C_{A,\text{uit}} (C_{A0} - C_{A,\text{uit}})}
    \]
\end{feedback}
\end{question}


\begin{question}
    Evaluate the following:
    \begin{enumerate}
        \item $|2-5| = \answer{3}$
        \item $|5-2| = \answer{3}$
        \item $|5-\sqrt{2}| = \answer{5-\sqrt{2}}$
        \item $|5-\sqrt{2}| = \answer{3.58578643763}$
    \end{enumerate}
    If $x > 3$ then:
    \begin{enumerate}
        \item $|x - 3| = \answer{x - 3}$
        \item $|3 - x| = \answer{x - 3}$
        \item $|\sin(x) - 3| = \answer{3 - \sin(x)}$
    \end{enumerate}
\end{question}
\end{exercise}


\end{document}