\documentclass{ximera}

\title{Theory: combinaties van reactoren}
\author{Mila Vervoort}
\license{CC: 0}         % replace with an appropriate license, or set it in xmPreamble

\begin{document}
\begin{abstract}
    Samenvatting van de theorie over speciale homogene reactoren onder isotherme voorwaarden.
\end{abstract}
\maketitle
\label{xim:theory:speciale reactoren}

\section{PFR met recyclestroom}
We beschouwen een plug flow reactor (PFR) met een externe recyclestroom.
De verse voeding heeft volumestroom $q$ en concentratie $C_{A0}$.
Een deel van de uitstroom wordt teruggevoerd naar de ingang van de reactor.

\begin{image}
\includegraphics[width=5cm]{PFR_met_recycle.png}
\end{image} 

\begin{itemize}
\item $q$ = volumestroom van de verse voeding
\item $R$ = recycleverhouding (dimensionloos)
\item Recycledebiet = $R q$
\item Totaal debiet door de reactor = $q (R+1)$
\item $C_{A,in}$ = concentratie aan de reactorinlaat
\item $C_{A,uit}$ = concentratie aan de reactoruitlaat
\item $V$ = reactorvolume
\end{itemize}

We nemen aan:
\begin{itemize}
\item constante dichtheid
\item stationaire toestand
\end{itemize}

De verse stroom en de recyclestroom mengen perfect vóór de reactor.
Daaruit volgt:
\[
q C_{A0} + R q C_{A,uit} = q (R+1) C_{A,in}
\]
Hieruit volgt:
\[
C_{A,in} = \frac{C_{A0} + R C_{A,uit}}{R+1}
\]
De inlaatconcentratie ligt dus tussen $C_{A0}$ en $C_{A,uit}$.\\

Vervolgens kunnen we voor deze reactor ook de stofbalans opstellen. 
\[
\text{Accumulation} = \text{In} - \text{Uit} + \text{Vorming} - \text{Verbruik}
\]
Voor een differentieel volume-element $dV$ geldt bij stationaire toestand:
\[
q(R+1)C_A - q(R+1)(C_A + dC_A) + r_A dV = 0
\]

Vereenvoudigen geeft:
\[
q(R+1)\, dC_A = r_A \, dV
\]
of
\[
\frac{dV}{q} = (R+1)\frac{dC_A}{-r_A}
\]
Integreren van $C_{A,in}$ tot $C_{A,uit}$ levert:
\[
\tau = \frac{V}{q} 
= (R+1)\int_{C_{A,uit}}^{C_{A,in}} \frac{dC_A}{-r_A}
\]

\begin{remark}
Fysische interpretatie 

De recycleverhouding R kan variëren van $0$ tot $\infty$
\begin{itemize}
\item Als $R = 0$:
\[
C_{A,in} = C_{A0}
\]
Het systeem reduceert tot een gewone PFR.
\item Als $R \rightarrow \infty$:
\[
C_{A,in} \rightarrow C_{A,uit}
\]
Er treedt sterke menging op en het gedrag benadert dat van een CSTR.
\end{itemize}
\end{remark}

\subsection*{Effect van recycle}
Voor een isotherm proces met monotoon stijgende kinetiek geldt dat de reactiesnelheid afneemt met dalende concentratie. In dat geval is een ideale PFR ($R=0$) het meest volume-efficiënt.\\

Het toevoegen van recycle verhoogt het totale debiet door de reactor met een factor $(R+1)$ en verlaagt de inlaatconcentratie. Hierdoor wordt de gemiddelde reactiesnelheid kleiner en neemt het benodigde volume om dezelfde conversie te behalen toe. In de limiet $R \rightarrow \infty$ wordt het vereiste volume gelijk aan dat van een CSTR.\\

\begin{image}
\includegraphics[width=5cm]{Volume_recycle.png}
\end{image} 

Recycle is dus voor zuiver isotherme systemen met eenvoudige, 
monotoon stijgende kinetiek niet gunstig vanuit volume-oogpunt. 
Bij niet-isotherme processen of complexere kinetiek kan recycle 
echter wél voordelig zijn.

\end{document}