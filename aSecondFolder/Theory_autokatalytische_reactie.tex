\documentclass{ximera}

\title{Autokatalytische reactie}
\author{Mila Vervoort}
\license{CC: 0}         % replace with an appropriate license, or set it in xmPreamble

\begin{document}
\begin{abstract}
    A simple theory explanation about autokatalytic reactions. %%  A one line description of the activity.
\end{abstract}
\maketitle
\label{xim:autokatalytische reactie}

\
\section*{Autokatalytische Reacties}

Een \textbf{autokatalytische reactie} is een chemische reactie waarbij één van de producten de reactie versnelt door als katalysator te fungeren. Met andere woorden, de reactie “katalyseert zichzelf”.  

Een typisch voorbeeld van een autokatalytische reactie is:

\[A \rightarrow B\]

Hierbij is \(B\) zowel een product als een katalysator voor de reactie. De reactiesnelheid neemt toe naarmate er meer \(B\) gevormd wordt.  

\textbf{Belangrijkste eigenschappen:}
\begin{itemize}
    \item De snelheid van de reactie neemt toe naarmate de concentratie van het product \(B\) toeneemt.
    \item Het kan leiden tot een \emph{sigmoidale} concentratieverloop in de tijd.
    \item Autokatalytische reacties komen vaak voor in biologische systemen en bepaalde chemische oscillatorreacties.
\end{itemize}

\textbf{Voorbeeld van snelheidsvergelijking:}

\[
-r_A = k C_A C_B
\]

waarbij:
\begin{itemize}
    \item \(C_A\) = concentratie van reactant \(A\)  
    \item \(C_B\) = concentratie van product/katalysator \(B\)  
    \item \(k\) = snelheidsconstante
\end{itemize}

\begin{center}
\geogebra{vrjscbe6}{700}{700}
\end{center}
\end{document}